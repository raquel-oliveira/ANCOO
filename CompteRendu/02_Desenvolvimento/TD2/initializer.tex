\textbf{Cas d'utilisation:}

\textbf{Acteur primaire:}

\textbf{Acteur support:}

\textbf{Pré-condition: } 
 
\textbf{Post-condition: } 

\textbf{Scenario primaire: } \\
    \textbf{1.} Le poste de surveillance ouvre une voie \\
    \textbf{2.} Le poste de surveillance initialise le bandeau d’information \\
    \textbf{3.} “Le poste de surveillance ouvre la barrière aval”

\textbf{Variantes:}\\
    \textbf{1a.} Le traffic n’est pas assez dense : le poste de surveillance ne l’ouvre pas
	l’usager ne peut pas passer \\
    \textbf{2a.} La borne de paiement est de type automatique : apparition de l’indicateur sur le bandeau\\
    \textbf{2b.} La borne de paiement est de type manuel : apparition de l’indicateur correspondant sur le bandeau\\
    \textbf{2c.} La borne est de type “abonnement” : apparition de l’indicateur correspondant \\
    \textbf{4a.} l’usager ne choisis pas la voie adapté \\
    \textbf{4 2a.} l’usager se trompe de voie : une intervention humaine est nécessaire pour le faire passer.\\