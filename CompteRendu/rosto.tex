%
% ********** Pagina de Rosto
%

% titlepage gera paginas sem numeracao
\begin{titlepage}

\begin{center}

\small

% O comando @{} no ambiente tabular x é  para criar um novo delimitador
% entre colunas que não a barra vertical | que é normalmente utilizada.
% O delimitador desejado vai entre as chaves. No exemplo, não há nada,
% de modo que o delimitador é vazio. Este recurso está sendo usado para
% eliminar o espaço que geralmente existe entre as colunas
\begin{tabularx}{\linewidth}{ c X }
% A figura foi colocada dentro de um parbox para que fique verticalmente
% centralizada em relação ao resto da linha
\parbox[c]{7cm}{\includegraphics[width=\linewidth]{polytech_logo}} &
\begin{center}
\textsf{\textsc{Polytech Nice Sophia\\ Analyse Conception Object
}} 
\end{center}

\end{tabularx}


% O vfill � um espa�o vertical que assume a m�xima dimens�o poss�vel
% Os vfill's desta p�gina foram utilizados para que o texto ocupe
% toda a folha
\vfill

\LARGE

\textbf{Modélisation object (version 2)}

\vfill

\Large

\textbf{\href{mailto:gabicavalcantesilva@gmail.com}{CAVALCANTE DA SILVA Gabriela},\\
\href{mailto:cesar.colle@gmail.com }{COLLE César},\\\href{mailto:oliveira.raquel.lopes@gmail.com}LOPES DE OLIVEIRA,\\ \href{mailto:arnold.schweitzer@gmail.com}{SCHWEITZER Arnold}}

\vfill

\normalsize

Enseignant: Colette Michel

\vfill

\hfill
\parbox{0.5\linewidth}{\textbf{
A faire:} Diagranne UC (he haute niveau); Diagramme UC détailllés et Scéncarios Cockburn correspondant.}


\vfill

\large

%Data
Valbonne, FR, février 26

\end{center}

\end{titlepage}
