% Pacotes Principais -----------------------------------------------------------
\usepackage[portuges,brazil]{babel}
\usepackage{ae} 
%%\usepackage[utf8]{inputenc}
%\usepackage[latin1]{inputenc}

%NO MAC ao inv�s de latin1 deve-se usar o applemac
\usepackage[applemac]{inputenc}

% Formatacaoo de capitulos ------------------------------------------------------
%\usepackage{capitulos}

% Figuras e Imagens ------------------------------------------------------------
\usepackage{graphicx}
% Figuras lado a lado
%\usepackage{epsfig} - REMOVIDO
%\usepackage{subfigure} - REMOVIDO

% Para poder ter tabelas com colunas de largura auto-ajust�vel
\usepackage{tabularx}

% Utilizar H para inserir as imagens REALMENTE onde eu desejo
%\usepackage{float} - REMOVIDO

% Para que os primeiros par�grafos das se��es tamb�m sejam indentados
% \usepackage{indentfirst} - REMOVIDO

%Permitir usar o comando \citeasnoun que coloca o nome na cita��o
\usepackage{harvard}


% Fontes -----------------------------------------------------------------------
%\usepackage[T1]{fontenc} - REMOVIDO
%\usepackage{pslatex} - REMOVIDO

% Matem�tico -------------------------------------------------------------------
%\usepackage{amsmath} - REMOVIDO
%\usepackage{amstext} - REMOVIDO

% Simbolos ---------------------------------------------------------------------
%\usepackage{textcomp} - REMOVIDO

% Tabelas ----------------------------------------------------------------------
%\usepackage{multirow} - REMOVIDO
% Colorir a tabela
%\usepackage{colortbl} - REMOVIDO

% Gloss�rio --------------------------------------------------------------------
%\usepackage[portuguese,noprefix]{nomencl}
%\usepackage{makeglo}

% Outros pacotes ---------------------------------------------------------------
%\usepackage{noitemsep} - REMOVIDO

%Subsubsection
%\usepackage{chngcntr} - REMOVIDO

% Para permitir espa�amento simples, 1 1/2 e duplo
\usepackage{setspace}

% Para executar um comando depois do fim da p�gina corrente
%\usepackage{afterpage} - REMOVIDO

% Para poder incluir arquivos Postscript com cores (do Xfig, por exemplo)
%\usepackage{color} - REMOVIDO

  % Itens numerados
%\usepackage{enumerate} - REMOVIDO

% Coment�rios em bloco
%\usepackage{verbatim} - REMOVIDO

% Refer�ncias ------------------------------------------------------------------
%\usepackage{html} - REMOVIDO
%\usepackage{url} - REMOVIDO
%\usepackage[abbr]{harvard}	% As chamadas s�o sempre abreviadas - REMOVIDO
%\harvardparenthesis{square}	% Colchetes nas chamadas - REMOVIDO
%\harvardyearparenthesis{round}	% Par�ntesis nos anos das refer�ncias - REMOVIDO
%\renewcommand{\harvardand}{e}	% Substituir "&" por "e" nas refer�ncias
%\renewcommand{\harvardurl}{URL: \url} - REMOVIDO